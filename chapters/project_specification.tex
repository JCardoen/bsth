\section{Project Description}
The emphasis of the \acrshort{poc} is on developing it in such a way that it should be easy to re-implement the application elsewhere. The PoC is developed in the two current formats for mobile development: iOS and Android. This bachelor's thesis will cover the implementation of the Android architecture. Firstly the existing application is reworked from using the Ionic framework to a native mobile application (Swift for iOS and Kotlin for Android). In addition to this part, geolocalization is implemented in the native mobile app using the MapWize service \cite{MapWize.io2019} and the IndoorLocation framework \cite{IndoorLocation.io2019}, both service provide working \acrfull{sdk} for iOS and Android. Finally the application is revised by the team of interns and the developers at IBM and uploaded onto the Apple Store and the Google Play Store.
\subsection{Technical Design Specs}
The communication with the hospital happens with a server provided by IBM and the hospital's \acrfull{api}. This means that the mobile device interacts with a intermediary server from IBM which in its turn communicates with the \acrshort{api} of the hospital. This model is an example of a highly reusable architecture. If another hospital needs to be attached to the IBM server, only a small 'translator' for the endpoints of the additional hospital's \acrshort{api} needs to be created whilst the structure of the IBM server remains the same.
\subsection{Features}
The main features of the project are specified below \cite{medappspec}:
\begin{enumerate}
\item Login with hospital provided credentials;
\item Synchronization of appointments with the hospital;
\item Ability to set reminders for an appointment;
\item See the hospital's location (and venues) as well as contact details;
\item Allow geolocalization inside the hospital;
\item Provide feedback after an appointment;
\item Localization in French, English and Dutch;
\item Available on both iOS and Android
\item Distributed in the Apple Store and Google Play Store;
\end{enumerate}
\subsection{Detailed View of Features}
\subsubsection{Login and Registration Process}
Firstly a patient should create an account to login into the application at a hospital site (mandatory to provide some details). Afterwards the patient will be able to login to the application and has the option to execute the following actions:
\begin{itemize}
\item Change userId, password (using the IBM BlueMix API)
\item Set an additional authentication method: pincode, face id or fingerprint when launching the app
\end{itemize}
To register upon arrival at the hospital, two of the following options can be provided: unique cipher code or via scanning a QR-code (displayed at the reception of the hospital).
\subsubsection{Appointments}
Overview of appointments:
\begin{enumerate}
\item View the upcoming or past appointments
\item Go to the notification screen and view notifications
\item Search appointments
\item The first upcoming appointment of the patient
\end{enumerate}
Available user actions possible in the overview:
\begin{itemize}
\item View the upcoming or past appointments
\item Go to the notification screen and view notifications
\item Search appointments
\end{itemize}
Individual view of an appointment:
\begin{enumerate}
\item Link to the specific doctor that will handle the appointment
\item Location
\item Preparatory notes
\item Description
\item Status
\end{enumerate}
The following user actions should be available for the end user:
\begin{itemize}
\item Register upon arrival at the specified location
\item Request a cancellation of the appointment
\item Call the hospital site
\item  After registration, meaning the user is at the hospital site, the option to show the indoor route to the place of appointment
\item Information and link to the hospital site
\end{itemize}
To create an appointment the user will have to fill in a specific form and is given the option to select a hospital site, service that he/she requires and a specific doctor. Upon submitting the form the hospital will receive a request for an appointment and can notify the user of his or her confirmation.
\subsubsection{General Information}
Users of the application should be available to view any information available on doctors, services and other important facets of the hospital (for instance: a news feed featuring health tips, recipes and important news on the hospital). This includes possible contact options such are telephone and email possibility. For the doctors, this means the following information needs to be displayed accurately:
\begin{itemize}
\item List of all available doctors
\item Possibility to search and filter doctors by name, function and service
\item Display the languages a doctor is able to converse in
\item A timetable that indicates availability based on time and place
\item Contact information if available
\end{itemize}
Other than the doctor's information, the services that the hospital provides should be able to be consulted. This includes the following information:
\begin{itemize}
\item List of all available services
\item Possibility to search and filter on the hospital site that offers this service and a concrete body part that situates the service.
\item Ability to contact the hospital site providing this service, either via email or phone.
\item Important times of day: office hours and visiting hours
\end{itemize}
For the sites this is:
\begin{itemize}
\item List of all sites of a specific hospital
\item Map view (Google map) that shows a location using a marker
\item Ability to contact the hospital site via email or phone.
\item Link to the specific website of this site, if it is available.
\end{itemize}
\subsubsection{Geolocalization}
The indoor map of the hospital displays general points of interest, such are: restaurant, toilets, elevators, staircases and reception. The route from the reception to the location of the appointment will be displayed when the user requests it inside the detailed view of an appointment (after registration process).
\subsubsection{User Settings}
The following settings should be displayed inside the application:
\begin{itemize}
\item Notification settings
\item FAQ
\item Feedback on the application
\item Information about the application: disclaimer, version, developer(s), privacy agreement
\item Level of mobility
\item Authentication method
\item Profile: email, password, userId, phone
\item Doctors of the user (where he or she has or will have an appointment)
\item Ability to sign out of the application
\end{itemize}
\subsubsection{Ideas}
Ideas for the application that can be implemented in future releases:
\begin{enumerate}
\item Symptom checker;
\item Chatbot to schedule and handle appointments
\item Prescriptions for medicines that can be used in apothecaries
\item 3D scan of patient's limb so a doctor can detect possible fractures or other superficial problems
\item Emergency button that will display the nearest emergency exit
\item Data integration of smart devices to determine possible health risks
\item Set reminder for medicine intake
\item Home Assistant integration
\item Details of the medical records of a patient
\end{enumerate}
\subsection{Technologies to research}
Throughout the development of the PoC, several technologies are used, such are: Android \acrshort{sdk}, authentication, RoomDB for offline storage, IBM BlueMix \acrshort{api}, \acrfull{uml}, dependency injection, MapWize, IndoorLocation and Cisco CMX.
\section{Development Guidelines}
To attain uniformity in the codebase of iOS and Android a 'Development Guidelines' document is written, this document can be found as an appendix.
