\section{Cisco CMX as IPS}
One of the companies that jumped on the bandwagon of the implementation of \acrlong{ips} is Cisco. Over the years they have developed a dashboard to gain intelligence from user's position such as hotspots where users gather most frequently and heat maps of different routes \acrlong{mu}s take. The definition provided by Cisco, states: ``Cisco’s CMX solution allows venues to simultaneously provide users with highly personalized content, provide services to customers to increase the customer experience, and gain visibility into customer behavior in their venues. CMX detects in-venue Wi-Fi enabled devices, prompts customers to connect to the wireless network, and engages them with value-added content and offers.`` \cite[p.~24]{Hallock2015}
\subsection{Functionality of Cisco CMX}
As featured in the datasheet provided by Cisco itself, Cisco CMX components can be distinguished into three different components, being: indoor location, CMX Connect, CMX Analytics \cite{Ciscoa}.
\section{Location}
\subsection{Location Techniques}
\subsubsection{Proximity, Presence}
This technique is most feasible when dealing with outdoor positioning as it requires less \acrlong{ap}s but has less accuracy. According to the datasheet of Cisco \cite{Ciscoa}, the accuracy of this technique is limited to 10 to 30 metres. To calculate the user's position, the \acrshort{ap} with the strongest \acrlong{rssi} is picked as a correct representation of the \acrshort{mu}'s location.
\subsubsection{RSSI Triangulation}
\subsubsection{Hyperlocation}
\subsection{CMX Connect}
\subsection{CMX Analytics}
\section{Performance Metrics}
\subsection{Accuracy ~\& Precision}
\subsection{Coverage Area}
\subsection{Scalability}
According to documents provided by Cisco, most of the WLAN controllers are scalable with a decent throughput. Some statistics:
\begin{enumerate}
\item Cisco 5508: up to 500 \acrshort{ap}s and 7,000 clients are supported with a throughput of 8\acrfull{gbps}
\item Cisco 7510: up to 6,000 \acrshort{ap}s and 64,000 clients are supported with a throughput of 1\acrfull{gbps} in centrally switched traffic \footnote{All WLAN traffic is switched through a user-only WLAN network, meaning a locally switched WLAN can be active for employees without interfering with the user-only network. ~\cite{Woland}}
\end{enumerate}
\subsection{Cost}
\subsection{Privacy}
\subsection{Conclusion}
\section{Cisco CMX Configuration}
\subsection{Case Study}
\subsection{Configuration Metric}